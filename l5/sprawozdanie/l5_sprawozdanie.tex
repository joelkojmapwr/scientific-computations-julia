\documentclass[12pt]{article}

\usepackage{graphicx} % Required for inserting images
\usepackage[utf8]{inputenc}
\usepackage[T1]{fontenc}
\usepackage{lmodern}
\usepackage[polish]{babel} % Używamy języka polskiego
\usepackage{amsmath}       % Do symboli matematycznych
\usepackage{graphicx}
\usepackage{booktabs}      % Do ładnych tabel (\toprule, \midrule, \bottomrule)
\usepackage{pdflscape}
\usepackage{multicol}

\usepackage{geometry}      % Lepsze marginesy
\geometry{a4paper, margin=1.5cm}

\title{Obliczenia Naukowe lista 5 - Sprawozdanie}
\author{Joel Kojma}
\date{\today}

\begin{document}

\maketitle

\section*{Pomysł na implementację eliminacji Gaussa dla macierzy rzadkich}
\begin{enumerate}
    \item Będziemy przechowywać macierz w postaci listy wektorów. Każdy wektor będzie odpowiadał jednemu wierszowi macierzy i będzie zawierał wartości od pierwszego niezerowego elementu w wierszu do ostatniego niezerowego elementu w wierszu. W ten sposób przechowujemy tylko niezerowe elementy macierzy. Dodatkowo będziemy używać tablicy offsetów (przesunięć), która mówi ile początkowych zer jest w każdym wierszu. Również będziemy mieli wektor długości wierszy - przechowuje długość tablicy.
\end{enumerate}


\end{document}