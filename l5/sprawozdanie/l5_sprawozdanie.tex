\documentclass[12pt]{article}

\usepackage{graphicx} % Required for inserting images
\usepackage[utf8]{inputenc}
\usepackage[T1]{fontenc}
\usepackage{lmodern}
\usepackage[polish]{babel} % Używamy języka polskiego
\usepackage{amsmath}       % Do symboli matematycznych
\usepackage{graphicx}
\usepackage{booktabs}      % Do ładnych tabel (\toprule, \midrule, \bottomrule)
\usepackage{pdflscape}
\usepackage{multicol}

\usepackage{geometry}      % Lepsze marginesy
\geometry{a4paper, margin=1.5cm}

\title{Obliczenia Naukowe lista 5 - Sprawozdanie}
\author{Joel Kojma}
\date{\today}

\begin{document}

\maketitle

\section*{Pomysł na implementację eliminacji Gaussa dla macierzy rzadkich}
\begin{enumerate}
    \item Będziemy przechowywać macierz w postaci listy wektorów. Każdy wektor będzie odpowiadał jednemu wierszowi macierzy i będzie zawierał wartości od pierwszego niezerowego elementu w wierszu do ostatniego niezerowego elementu w wierszu. W ten sposób przechowujemy tylko niezerowe elementy macierzy. Dodatkowo będziemy używać tablicy offsetów (przesunięć), która mówi ile początkowych zer jest w każdym wierszu. Również będziemy mieli wektor długości wierszy - przechowuje długość tablicy.
\end{enumerate}

\section*{Złożoność pamięciowa}
\begin{enumerate}
    \item Tablice \texttt{row\_offsets} i \texttt{row\_lengths} służące do przeliczania indeksów mają rozmiar $n$.
    \item W każdym wierszu pamiętamy elementy macierzy $A_k$, których jest dokładnie $l$, z macierzy $C_k$ pamiętamy aż do elementu niezerowego na przekątnej, więc średnio jest to $l/2$ elementów na wiersz. W macierzy $B_k$ pamiętamy dokładnie średnio na wiersz $\left(\dfrac{2l-1}{l^2}\right) \simeq \frac{2}{l}$ elementów. \\ Zatem złożoność pamięciowa naszej implementacji eliminacji Gaussa wynosi: $O(n \cdot l)$, więc jeśli $l$ jest stałe, to złożoność pamięciowa jest liniowa względem rozmiaru macierzy $n$.
\end{enumerate}

\section*{Złożoność Obliczeniowa}
\begin{enumerate}
    \item Eliminacja Gaussa: Doprowadzenie do postaci macierzy trójkątnej górnej:
    \begin{itemize}
        \item Przechodzimy po wszystkich kolumnach macierzy ($n$ kolumn)
        \item Dla każdej kolumny przechodzimy przez wszystkie wiersze poniżej przekątnej (z własności macierzy jest ich nie więcej niż $l$). W każdym wierszu wykonujemy eliminację gaussa. Mnożymy nie więcej niż $2l$ razy, ponieważ nie musimy mnożyć przez elementy równe zero ponieważ nic nie zmieniają.
        \item Zatem złożoność obliczeniowa eliminacji Gaussa wynosi $O(n \cdot l \cdot 2l) = O(n \cdot l^2)$, więc jeśli $l$ jest stałe, to złożoność obliczeniowa jest liniowa względem rozmiaru macierzy $n$.
        \item Dla wariantu z częściowym wyborem elementu głównego, dodatkowo w każdej kolumnie musimy znaleźć maksymalny element w kolumnie spośród nie więcej niż $l$ elementów, lecz dla stałego $l$ złożność dalej pozostaje $O(n)$ liniowa.
    \end{itemize}
    \item Znalezienie rozwiązania wektora $x$ z macierzy trójkątnej górnej:
    \begin{itemize}
        \item W każdym wierszu wykonujemy sumę nie więcej niż $2l$ elementów.
        \item Zatem złożoność obliczeniowa wynosi $O(n \cdot 2l) = O(n \cdot l)$, więc jeśli $l$ jest stałe, to złożoność obliczeniowa jest liniowa względem rozmiaru macierzy $n$.
    \end{itemize}
\end{enumerate}

\end{document}