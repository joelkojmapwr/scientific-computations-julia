\documentclass[12pt]{article}

\usepackage{graphicx} % Required for inserting images
\usepackage[utf8]{inputenc}
\usepackage[T1]{fontenc}
\usepackage{lmodern}
\usepackage[polish]{babel} % Używamy języka polskiego
\usepackage{amsmath}       % Do symboli matematycznych
\usepackage{graphicx}
\usepackage{booktabs}      % Do ładnych tabel (\toprule, \midrule, \bottomrule)
\usepackage{pdflscape}
\usepackage{multicol}

\usepackage{geometry}      % Lepsze marginesy
\geometry{a4paper, margin=1.5cm}

\title{Obliczenia Naukowe lista 4 - Sprawozdanie}
\author{Joel Kojma}
\date{\today}

\begin{document}

\maketitle

\section*{Zad 1}
Obliczam ilorazy różnicowe, według wzorów podanych na wykładzie. W każdej iteracji trzymam tylko aktualy rząd ilorazów różnicowych -> Unikam tablicy dwuwymiarowej.\\
W iteracji $k$ zapisuję do tablicy wynikowej wartość w pierwszej komórce tablicy ilorazów różnicowych -> $f[x_0, x_1, \dots , x_k]$\\

\section*{Zad 2}
Algorytm został zaimplementowany tak jak jest podany w Zadaniu 8 Listy 4 z ćwiczeń.\\

\section*{Zad 3}
Mamy wielomian w postaci Newtona:
$$P(x) = c_0 + c_1(x-x_0) + c_2(x-x_0)(x-x_1) + \ldots + c_n(x-x_0)(x-x_1)\cdots(x-x_{n-1})$$

gdzie $c_i = f[x_0, x_1, \ldots, x_i]$ to ilorazy różnicowe.\\
Wiem, że algorytm na rozwiązanie bierze się z tej postaci Newtona i że możnaby ją przekształcić do postaci Hornera, ale nie udało mi się dokładnie zrozumieć skąd się to bierze.

\section*{Zad 4}
Pierwiastki Czebyszewa stopnia n wyliczamy ze wzoru: 
$$x_j = \cos\left(\frac{(2j + 1)\pi}{2n}\right), \quad j = 0, 1, \ldots, n-1$$

A następnie przeskalowujemy je do przedziału $[a, b]$ ze wzoru:
$$x_j' = \frac{a + b}{2} + \frac{b - a}{2} x_j$$

Wykres interpolowanej funkcji rysujemy na 1000 równomiernie rozmieszczonych punktach w przedziale $[a, b]$. Licząc wartość funkcji w punkcie używając funkcji warNewton.

\section*{Testy funkcji 1-4}
Funkcje zostały przetestowane dla funkcji $f(x) = x^2$ i zwracają prawidłowe wyniki.

\section*{Zad 5}
Obie interpolowane funkcje są bardzo dokładnie odwzorowane dla wszystkich $n=5, 10, 15$

\section*{Zad 6}

Zarówno dla funkcji $f(x) = |x|$ jak i $g(x) = \frac{1}{1 + x^2}$, gdy wybierzemy punkty równoodległe interpolacja działa poprawnie dla małych $n$, ale dla większych $n$ pojawiają się duże oscylacje przy krańcach przedziału (efekt Rungego).\\ 

Natomiast gdy wybierzemy pierwiastki Czebyszewa jako punkty interpolacji, to nawet dla większych $n$ interpolacja działa bardzo dobrze i nie pojawiają się oscylacje przy krańcach przedziału.Funkcja staje się coraz dokładniej odwzorowywana wraz ze wzrostem $n$.

\begin{figure}[h]
    \centering
    \includegraphics[width=0.8\textwidth]{images/abs_5_rownoodlegle.png}
    \caption{Interpolacja funkcji $f(x) = |x|$ z węzłami równoodległymi dla $n=5$}
    \label{fig:abs_5_rownoodlegle}
\end{figure}

\begin{figure}[h]
    \centering
    \includegraphics[width=0.8\textwidth]{images/abs_15_rownoodlegle.png}
    \caption{Interpolacja funkcji $f(x) = |x|$ z węzłami równoodległymi dla $n=15$}
    \label{fig:abs_15_rownoodlegle}
\end{figure}

\begin{figure}[h]
    \centering
    \includegraphics[width=0.8\textwidth]{images/abs_5_czebyszew.png}
    \caption{Interpolacja funkcji $f(x) = |x|$ z węzłami będącymi pierwiastkami Czebyszewa dla $n=5$}
    \label{fig:abs_5_czebyszew}
\end{figure}

\begin{figure}[h]
    \centering
    \includegraphics[width=0.8\textwidth]{images/abs_15_czebyszew.png}
    \caption{Interpolacja funkcji $f(x) = |x|$ z węzłami będącymi pierwiastkami Czebyszewa dla $n=15$}
    \label{fig:abs_15_czebyszew}
\end{figure}
\end{document}