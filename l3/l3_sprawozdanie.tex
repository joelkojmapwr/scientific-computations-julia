\documentclass[12pt]{article}

\usepackage{graphicx} % Required for inserting images
\usepackage[utf8]{inputenc}
\usepackage[T1]{fontenc}
\usepackage{lmodern}
\usepackage[polish]{babel} % Używamy języka polskiego
\usepackage{amsmath}       % Do symboli matematycznych
\usepackage{graphicx}
\usepackage{booktabs}      % Do ładnych tabel (\toprule, \midrule, \bottomrule)
\usepackage{pdflscape}
\usepackage{multicol}

\usepackage{geometry}      % Lepsze marginesy
\geometry{a4paper, margin=2.5cm}

\title{Obliczenia Naukowe lista 3 - Sprawozdanie}
\author{Joel Kojma}
\date{\today}

\begin{document}

\maketitle

\section*{Zad 1}
\subsection*{Intuicja}
Będziemy szukać rozwiązania metodą bisekcji.
Zamiast rekurencyjnie spróbuję wykonać zadanie iteracyjnie, aby zminimalizować liczbę wywołań funkcji.

\section*{Zad 3}
W metodzie siecznych musimy uważać na dzielenie przez zero.
Czyli sytuację, kiedy $f(a) - f(b) = 0 \rightarrow$ Wtedy prosta wogóle nie przecina osi $X$. 

\end{document}