\documentclass[12pt]{article}

\usepackage{graphicx} % Required for inserting images
\usepackage[utf8]{inputenc}
\usepackage[T1]{fontenc}
\usepackage{lmodern}
\usepackage[polish]{babel} % Używamy języka polskiego
\usepackage{amsmath}       % Do symboli matematycznych
\usepackage{graphicx}
\usepackage{booktabs}      % Do ładnych tabel (\toprule, \midrule, \bottomrule)
\usepackage{pdflscape}
\usepackage{multicol}

\usepackage{geometry}      % Lepsze marginesy
\geometry{a4paper, margin=1.5cm}

\title{Obliczenia Naukowe lista 3 - Sprawozdanie}
\author{Joel Kojma}
\date{\today}

\begin{document}

\maketitle

\section*{Zad 1}
\subsection*{Intuicja}
Będziemy szukać rozwiązania metodą bisekcji.
Zamiast rekurencyjnie spróbuję wykonać zadanie iteracyjnie, aby zminimalizować liczbę wywołań funkcji.

\section*{Zad 2}

\subsection*{Działanie algorytmu metody siecznych}
Zaczynamy od punktu początkowego $x_0$. Liczymy pochodną funkcji $f$ w tym punkcie i na jej podstawie wyliczamy styczną do funkcji w punkcie $x_0$.
Następnie wyznaczamy punkt przecięcia tej stycznej z osią $X$ i oznaczamy go jako $x_1$.
Następnie powtarzamy to samo dla punktu $x_1$ i kolejnych $x_n$.

\section*{Zad 3}
W metodzie siecznych musimy uważać na dzielenie przez zero.
Czyli sytuację, kiedy $f(a) - f(b) = 0 \rightarrow$ Wtedy prosta wogóle nie przecina osi $X$. 

\section*{Moje testy do zadań 1-3}
\subsection*{Wielomian $f(x) = x^3 - x - 2$}
Testowałem dla parametrów: 
$\delta = 10^{-5}, \epsilon = 10^{-5}, maxit = 100$.
\subsubsection*{Wyniki}
\begin{table}[h]
\centering
\begin{tabular}{ccccc}
\toprule
algorytm & $r$ & $f(r)$ & iter & błąd względny [\%] \\
\midrule
m. bisekcji & $1.52139 \times 10^{0}$ & $3.26082 \times 10^{-5}$ & 17 & $3.605981 \times 10^{-4}$ \\
m. stycznych & $1.52138 \times 10^{0}$ & $-1.53692 \times 10^{-6}$ & 7 & $1.699610 \times 10^{-5}$ \\
m. siecznych & $1.52138 \times 10^{0}$ & $-1.64365 \times 10^{-7}$ & 6 & $1.817644 \times 10^{-6}$ \\
\bottomrule
\end{tabular}
\caption{Wyniki dla wielomianu $f(x) = x^3 - x - 2$}
\end{table}

\section*{Zad 4}
Funkcja $f(x) = \sin(x) - 0.25x^2$

\begin{table}[h]
\centering
\begin{tabular}{ccccc}
\toprule
algorytm & $r$ & $f(r)$ & iter & błąd względny [\%] \\
\midrule
m. bisekcji & $1.93375 \times 10^{0}$ & $-2.70277 \times 10^{-7}$ & 16 & $1.057312 \times 10^{-5}$ \\
m. stycznych & $1.93375 \times 10^{0}$ & $-2.24233 \times 10^{-8}$ & 4 & $8.771914 \times 10^{-7}$ \\
m. siecznych & $1.93375 \times 10^{0}$ & $1.56453 \times 10^{-7}$ & 4 & $6.120361 \times 10^{-6}$ \\
\bottomrule
\end{tabular}
\caption{Wyniki dla funkcji $f(x) = \sin(x) - 0.25x^2$}
\end{table}

\subsection*{Obserwacje}
Metoda stycznych i siecznych są 4 razy szybsze od metody bisekcji.
Podobnie wyszło również dla moich testów.
Metoda siecznych jest szybsza od metody bisekcji nawet pomimo tego, że ma dwa razy większy przedział początkowy.

\subsection*{Wniosek}
Jeśli chcemy znaleźć miejsca zerowe funkcji na dużym przedziale w szybki sposób, to najlepiej używać metody stycznych albo siecznych.

\section*{Zad 5}
\subsection*{Plan rozwiązania}
Chcemy znaleźć wartości $x$ dla których przecinają się funkcje $f(x) = 3x$ i $g(x) = e^x$
\begin{align*}
    \begin{cases}
        y &= 3x \\
        y &= e^x
    \end{cases}
    \Rightarrow 3x = e^x \Rightarrow e^x - 3x = 0
\end{align*}
Więc aby rozwiązać zadanie, wystarczy znaleźć miejsca zerowe funkcji $h(x) = e^x - 3x$.
Miejsca zerowe tej funkcji znajdują się w przedziałach $[0, 1]$ oraz $[1, 2]$.
Dla tych przedziałów użyjemy metody bisekcji.

\subsection*{Wyniki}
$r_1 = 0.619140625 \rightarrow $ Algorytm wykonał 9 iteracji \\
$r_2 = 1.51171875 \rightarrow $ Algorytm wykonał 8 iteracji \\

\section*{Zad 6}
\subsection*{Funkcja $f_1(x) = e^{1-x}-1$}
\subsubsection*{Parametry wywołania}
Łatwo można sprawdzić, że $f_1(0) > 0$ oraz $f_1(2) < 0$, więc miejsce zerowe znajduje się w przedziale $[0, 2]$. Miejsce zerowe znajduje się dokładnie w punkcie $x=1$.
Dla bisekcji dobieramy przedział początkowy $[0, 3]$. Gdybym wziął $[0, 2]$, to algorytm wykonałby tylko 1 iterację (szczęśliwy przypadek).
Dla metody stycznych wybieram punkt startowy $1.5$.
Dla metody siecznych wybieram punkty początkowe $1.5$ oraz $2.0$.

\subsubsection*{Wyniki}
\begin{table}[h]
\centering
\begin{tabular}{cccc}
\toprule
algorytm & $r$ & $f(r)$ & iter \\
\midrule
m. bisekcji & $9.99996 \times 10^{-1}$ & $3.81470 \times 10^{-6}$ & 18 \\
m. stycznych & $1.00000 \times 10^{0}$ & $1.52638 \times 10^{-9}$ & 4 \\
m. siecznych & $1.00000 \times 10^{0}$ & $-3.42698 \times 10^{-6}$ & 5 \\
\bottomrule
\end{tabular}
\caption{Wyniki dla funkcji $f_1(x) = e^{1-x}-1$}
\end{table}

\subsubsection*{Eksperyment w metodzie Newtona $x_0 \in (1, \infty)$}
Dla punktów startowych większych od 5, liczba iteracji zaczyna rosnąć bardzo gwałtownie. Wynika to z faktu, że funkcja się zaczyna wypłaszczać, wobec tego trzeba tworzyć coraz więcej stycznych, a wartości do których doszedł algorytm są ujemne.

\subsection*{Funkcja $f_2(x) = xe^{-x}$}
Łatwo sprawdzić, że $f_2(0) = 0$ więc to jest miejsce zerowe, które musimy znaleźć. 
Funkcja w nieskończoności dąży do zera. 
\subsubsection*{Parametry wywołania}
Dla bisekcji wybieram przedział początkowy $[-3.23, 1.433]$.
Dla metody stycznych wybieram punkt startowy $0.5$.
Dla metody siecznych wybieram punkty początkowe $0.5$ oraz $0.8$.

\subsubsection*{Wyniki}

\begin{table}[h]
\centering
\begin{tabular}{cccc}
\toprule
algorytm & $r$ & $f(r)$ & iter \\
\midrule
m. bisekcji & $4.08936 \times 10^{-6}$ & $4.08934 \times 10^{-6}$ & 14 \\
m. stycznych & $-3.06425 \times 10^{-7}$ & $-3.06425 \times 10^{-7}$ & 5 \\
m. siecznych & $-2.61460 \times 10^{-6}$ & $-2.61461 \times 10^{-6}$ & 8 \\
\bottomrule
\end{tabular}
\caption{Wyniki dla funkcji $f_2(x) = xe^{-x}$}
\end{table}

\subsubsection*{Obserwacje}
Najdokładniejsza jest metoda Newtona (stycznych).

\subsubsection*{Eksperyment w metodzie Newtona $x_0 \ge 1$}
Funkcja $f_2(x)$ ma maksimum lokalne w punkcie $x=1$. Dla $ x >1$ funkcja jest malejąca i zbiega do zera w nieskończoności. 
Metoda Newtona zamiast zbiegać do $x=0$, będzie zbiegać do nieskończoności, ponieważ tam funkcja dąży do zera, ale nigdy do niego nie dociera.
Podobnie dzieje się dla metody siecznych. \\
Dla $x=1.0$ algorytm nie wykona się, ponieważ pochodna w tym punkcie jest równa zero, więc musielibyśmy podzielić przez zero, więc funkcja zwraca błąd.

\subsubsection*{Wnioski}
Metody stycznych Newtona i siecznych są szybsze niż metoda bisekcji, ale nie gwarantują nam zbieżności do miejsca zerowego.


\end{document}