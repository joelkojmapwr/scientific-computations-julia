\documentclass[12pt]{article}

\usepackage{graphicx} % Required for inserting images
\usepackage[utf8]{inputenc}
\usepackage[T1]{fontenc}
\usepackage{lmodern}
\usepackage[polish]{babel} % Używamy języka polskiego
\usepackage{amsmath}       % Do symboli matematycznych
\usepackage{graphicx}
\usepackage{booktabs}      % Do ładnych tabel (\toprule, \midrule, \bottomrule)
\usepackage{pdflscape}
\usepackage{multicol}

\usepackage{geometry}      % Lepsze marginesy
\geometry{a4paper, margin=1.5cm}

\title{Obliczenia Naukowe lista 3 - Sprawozdanie}
\author{Joel Kojma}
\date{\today}

\begin{document}

\maketitle

\section*{Zad 1}
\subsection*{Intuicja}
Będziemy szukać rozwiązania metodą bisekcji.
Zamiast rekurencyjnie spróbuję wykonać zadanie iteracyjnie, aby zminimalizować liczbę wywołań funkcji.

\section*{Zad 3}
W metodzie siecznych musimy uważać na dzielenie przez zero.
Czyli sytuację, kiedy $f(a) - f(b) = 0 \rightarrow$ Wtedy prosta wogóle nie przecina osi $X$. 

\section*{Moje testy do zadań 1-3}
\subsection*{Wielomian $f(x) = x^3 - x - 2$}
Testowałem dla parametrów: 
$\delta = 10^{-5}, \epsilon = 10^{-5}, maxit = 100$.
\subsubsection*{Wyniki}
\begin{table}[h]
\centering
\begin{tabular}{ccccc}
\toprule
algorytm & $r$ & $f(r)$ & iter & błąd względny [\%] \\
\midrule
m. bisekcji & $1.52139 \times 10^{0}$ & $3.26082 \times 10^{-5}$ & 17 & $3.605981 \times 10^{-4}$ \\
m. stycznych & $1.52138 \times 10^{0}$ & $-1.53692 \times 10^{-6}$ & 7 & $1.699610 \times 10^{-5}$ \\
m. siecznych & $1.52138 \times 10^{0}$ & $-1.64365 \times 10^{-7}$ & 6 & $1.817644 \times 10^{-6}$ \\
\bottomrule
\end{tabular}
\caption{Wyniki dla wielomianu $f(x) = x^3 - x - 2$}
\end{table}

\end{document}